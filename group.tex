\documentclass[12pt]{book}
\usepackage{mathtools}
\usepackage{graphicx}
\usepackage[utf8]{inputenc}
\usepackage{float}
\usepackage{tabularx}
\usepackage{chngpage}
\usepackage{amsthm}
\usepackage{mathtools}
\usepackage{amsfonts}
\usepackage{tikz}
\usepackage{amssymb}

\usepackage{wrapfig}
\usepackage{enumerate}
\usepackage{braket}
\usepackage{bbm}
\usepackage{fancyhdr}
\usepackage{emptypage}

\usepackage{wrapfig,lipsum,booktabs}
\input{insbox.tex}


\usetikzlibrary{decorations.markings}

\theoremstyle{plain}

%\numberwithin{equation}{section}
\newcommand{\R}{\mathbb{R}}
\newcommand{\I}{\mathbbm{1}}
\newcommand{\Z}{\mathbb{Z}}
\newcommand{\F}{\mathcal{F}}
\renewcommand{\L}{\mathcal{L}}
\newcommand{\C}{\mathbb{C}}
\renewcommand{\H}{\mathcal{H}}
\newcommand{\Sum}{\sum_{n=0}^\infty}
\newcommand{\Res}[1]{\text{Res}f(z)\Big|_{#1}}
\newcommand{\vettore}[1]{\overrightarrow{#1}}
\newcommand{\p}{\mathbf{p}}
\newcommand{\x}{\mathbf{x}}


\newtheorem{thm}{Theorem}[section]
\newtheorem{prop}[thm]{Proposizione}
\newtheorem{coro}[thm]{Corollary}
\newtheorem{lem}[thm]{Lemma}
\theoremstyle{definition}
\newtheorem{dfn}[thm]{Definition}

\theoremstyle{remark}
\newtheorem*{rmk}{Remark}
\renewcommand{\d}[2]{\frac{d #1}{d #2}} % for derivatives
\newcommand{\dd}[2]{\frac{d^2 #1}{d #2^2}} % for double derivatives
\newcommand{\pd}[2]{\frac{\partial #1}{\partial #2}} 
% for partial derivatives
\newcommand{\pdd}[2]{\frac{\partial^2 #1}{\partial #2^2}} 
% for double partial derivatives

\title{\textbf{Group theory and matrix analysis}}
\author{Marco Canteri}
\date{2017}

\def\changemargin#1#2{\list{}{\rightmargin#2\leftmargin#1}\item[]}
\let\endchangemargin=\endlist


\usepackage[a4paper, inner=1.5cm, outer=3cm, top=3cm, 
bottom=3cm, bindingoffset=1cm,headheight=110pt]{geometry} 

\pagestyle{fancy}
\fancyhf{}
\fancyhead[LE]{\leftmark}
\fancyhead[RO]{\rightmark}
\fancyfoot[C]{\thepage}
\begin{document}
\maketitle
\tableofcontents
\chapter{Group theory and representation theory}
\section{Fundamentals of groups}
Group theory is a branch of mathematics that study \emph{Groups}, groups are useful, for instance, to describe symmetries in a system. In physics, symmetries have a central role, a problem can be simplified in presence of symmetries and some physical quantities are directly correlated to symmetries. Therefore, the study of group is indeed important in theoretical physics.
Let us start from the beginning by defining what is a group
\begin{dfn}
Let $G$ be a set and $u$ a map $u:G\times G\to G$, then the pair $(G,u)$ is called a \textbf{group} if the following conditions hold
\begin{enumerate}[i.]
\item (Associative) $g(fh) = (gf)h$;
\item (Neutral element) $\exists e\in G: eg=ge=g\quad\forall g\in G$;
\item (Inverse element) $\forall g\in G\,$ $\exists g^{-1}\in G: gg^{-1} = g^{-1}g=e$;
\end{enumerate}
notice that the map $u$ is usually represented like the standard multiplication $gh:=u(g,h)$ and very often we call $G$ a group instead of the pair $(G,u)$.
\end{dfn}
\begin{dfn}
A group $G$ is called \textbf{Abelian} if $gh=hg\quad \forall g,h\in G$.
\end{dfn}
A group can have several proprieties, one of the most important is the order of a group, which can be defined as follow
\begin{dfn}
The \textbf{order} of a group $G$ is the number of elements of $G$ (cardinality) and it is denoted as $|G|$. 
\end{dfn}
There are two cases, either the order of a group is finite (or countable infinite), and in this case the group is called discrete, or the order in not finite (uncountable infinite) and we say that the group is continuous. In order to clarify what is a group let us do some interesting examples of a group:
\begin{itemize}
\item $G = \Z_2 = \{-1,1\}$ and $u$ the standard multiplication.\\
 We need to check that this is a group, first of all the map $u$ must be closed with respect to the set, this is easy to show, in fact there are only two elements and with the standard multiplication we know that multiplying $-1$ for $1$ or vice versa will always end up with -1. The standard multiplication is also associative. The neutral element of this group is $1$ and the inverse of $g\in G$ in this group is $g$ itself. Therefore, this is a finite group.
\item $G = \{z\in \C : |z|=1\}$, $u$ standard multiplication.\\
This is also a group, in fact we can represent a complex number with a complex exponential $z=e^{i\theta}$, multiplying two complex number will only change the phase of the number, not the modulus, thus multiplication is closed. It is also associative, the neutral element is $1$ and the inverse element is the complex conjugate $zz^* = |z|^2 = 1$.
\item $G = \{e,a,a^2,a^3\}$ with $a^4=e$ and also $a^0=e$, with the map defined as $u(a^k,a^l) = a^{k+l}$.\\
It is closed, associative, the neutral element is $e$ and the inverses are: $aa^3 = e$, $a^2a^2=e$, $a^3a=e$. Therefore, this is a group of order 4.
\item $G = GL(n,\C) = \{A\in M_n(\C): \text{det}A\neq 0\}$ and $u$ is matrix multiplication. Note that $GL(n,\C)$ is usually called General Linear group and is the set of $n\times n$ invertible matrices.\\
$u$ is closed, associative, the neutral element is $\mathbbm{1}$ and the inverse is the inverse matrix $A^{-1}\in G$. Hence, this is a group.
\item $G = \{U\in M_n : U^{-1}U = \I\}$, i.e. the set of unitary matrices with the matrix multiplication.\\
The product of two unitary matrices is still unitary, so the map is closed. It is also associative, the neutral element is $\I$ and the inverse is $U^\dagger$. Therefore, it is a group.
\end{itemize}
A particular class of Group is the so called symmetric groups. Given $n$ elements $\overline{G}=\{1,2,\dots,n\}$, the symmetric group is defined as the set of permutations $\Pi:\overline{G}\to \overline{G}$, which we can write as $S_n = \{\Pi:\overline{G}\to \overline{G}: \text{where } \Pi \text{ is a bijective permutation}\}$. If $u$ is the concatenation of two permutations, then $(S_n,u)$ is a symmetric group. In fact $u$ is associative, the neutral element is the identity map, and the inverse element is $\Pi^{-1}$, which exist because permutations are bijective. Note also that the order of a symmetric group is $n!$, indeed there are $n!$ possible permutations in a set with $n$ elements. For instance, let us take $S_3$, we will denote the permutation as $\Pi_{1,2}\equiv (1,2)$, i.e the permutation that swap 1 with 2: $1\to 2,2\to 1,3\to3$, and $(1,2,3)$ as the permutation that changes every number like this: $1\to2,2\to3,3\to1$. Therefore the set $S_3$ is the following
\[S_3 = \{\I,(1,2),(2,3),(1,2,3),(1,3,2),(1,3)\}\]
there are of course $3!=6$ elements which correspond to the symmetry of a triangle, as we will see later in more detail.\\ 
We can also define a dihedral group $D_n,n\geq 3$ which denotes the orthogonal symmetries of a regular polygon, which has $n$ vertices and is centered in the origin. The vertices are denoted by $V_0,V_1,\dots,V_{n-1}\in\R^2$ in counter clockwise direction. The symmetries of a polygon are: $C_n = R_{\frac{2\pi}{n}}$, that is a counter clockwise rotation of angle $2\pi/n$; $\sigma^{(i)}$, that is a reflection with respect to the line through $V_i$ and the center. Therefore, we can write $D_n$ as
\[D_n = \{\I, C_n,C_n^2,\dots,C_n^{n-1},\sigma^{(i)},C_n\sigma^{(i)},\dots C_n^{n-1}\sigma^{(i)} \}\]
as can be easily seen, the order of this group is $|D_n| = 2n$. In the case $n=3$ we would have $D_3 = \{\I,C_3,C_3^2,\sigma^{(1)},C_3\sigma^{(1)},C_3\sigma^{(1)}\}$, notice that for instance $\sigma^{(3)} = C_3 \sigma^{(1)}$. Moreover, in this case we have that $S_3 = D_3$, in fact we can denotes every vertex of a triangle with a number and exchanging numbers is equivalent of rotating of reflecting the triangle.
\InsertBoxR{0}{
  \begin{minipage}{5.5cm}
\begin{tabular}{c|cccc}
G & $g_1$ & $g_2$ &  & $g_n$ \\
\hline 
$g_1$ &$g_1g_1$ & $g_1g_2$ &\dots& $g_1g_n$\\  
$g_2$ &$g_2g_1$ & $g_2g_2$ & & $g_2g_n$\\ 
\vdots\\
$g_n$ &$g_ng_1$ & $g_ng_2$ & & $g_ng_n$\\ 
\end{tabular}
\end{minipage}}[3]In the case of finite group, $u$ doesn't have to follow any algebraic rules. Hence, it is sometimes convenient to write $u(g_i,g_j)$ for $g_i,g_j\in G$ into a table. Due to the existence of the neutral element, there exist one row (column) which coincides with the first row (column). Furthermore, due to the existence of the inverse element, each row and each column contains all $n$ elements. In fact, suppose that there exist $i$ such that one element occurs at least twice in the $i$-th row, i.e. $\exists j\neq k : g_ig_j = g_ig_k$.  Then it also true that $g_i^{-1}g_ig_j = g_i^{-1}g_ig_k \implies g_j = g_k$ which is a contradiction. Therefore, all the $n$ elements of the $i$-th row must be different, that is each row contains all $n$ element of $G$. The same argument can be done with a column. From this we can say that the multiplication of $g_i$ corresponds to a permutation $\Pi_{g_i}$ of the $n$ elements of $G$.
\section{Subgroups}
A subgroup of a group is basically a subset of a group which is itself a group, or more formally
\begin{dfn}
Let $G$ be a group, then $H\subset G$ is a \textbf{subgroup} of $G$ if
\begin{enumerate}[i.]
	\item $e_g \in H$
	\item $\forall g,h \in H,$ $gh\in H$
	\item $\forall g\in H,$ $g^{-1}\in H$
\end{enumerate}
\end{dfn}
To distinguish between subgroup and subset, there is a different notation, so if $H$ is a subgroup of $G$ we write $H\leq G$, while if $H$ is only a subset the notation is the standard $H\subset G$. A couple of examples of subgroups, which are relevant for their applications in physics, are
\begin{itemize}
\item $SL(n,\C) = \{A\in GL(n,\C): |A| = 1\}\leq GL(n,\C)$, $SL$ is called the special linear group and it is a subgroup of the general linear group of matrices which have determinant equals to one.
\item $SO(n) = \{A\in SL(n,\R) : A^\dagger A = \I\}\leq SL(n,\R)\leq GL(n,\C)$ 
\item $U(n) = \{A\in GL(n,\C):A^\dagger A= \I \} \leq GL(n,\C)$
\end{itemize}
A property of subgroup is that if we take $H_i\leq G$ subgroups, with an index $i\in I$ the the intersection of the these subgroup is still a subgroup, that is
\[H = \bigcap_{i\in I} H_i \leq G.\]
The proof is straightforward, we only need to check if the intersection has the proprieties of a group. Associative propriety is obvious. Since $H_i$ are a group, the neutral element is in each of them, so the neutral element is also in the intersection. The same argument goes for the inverse, $g\in H_i \implies g^{-1}\in H_i$, which means that $\forall g\in H \implies g^{-1}\in H$. And lastly with the same logic as before if $g,h\in H_i\implies gh \in H_i$ then $gh\in H$ if $g,h\in H$.\\
As already said, subsets and subgroup are different, but it is also possible to generate a subgroup from a subset. Let $G$ be a group and $E\subset G$ a subset, then the subgroup generated by $E$ is
\[\braket{E} = \bigcap_{\substack{E\subseteq H\\H\leq G}} H,\]
which is the smallest subgroup that contains $E$. It can be easily computed by taking the elements of $E$, all their inverse and all their products. For instance, let us consider the set $E = \{C_3\}$, the inverse of this element is $C_3^2$ and the neutral element is $\I$. Therefore the subgroup generated by $E$ is
\[\braket{E} = \{C_3,C_3^2,\I\}.\]
It is also easy to show that the following symmetric group can be written as $S_3 = \braket{\{C_3,\sigma^{(1)}\}}$, or that $\Z$ can be generated by $\Z = \braket{\{1\}}=\braket{\{-1\}}$, where the map $u$ is the addition. Last example shows also that a generate set of a subgroup is not necessarily unique. Moreover it is called cyclic, the general definition is the following
\begin{dfn}
A group is called \textbf{cyclic} if it is generated by a single element.
\end{dfn}
Cyclic groups $G$ can be divided in two classes:
\begin{itemize}
	\item If $|G|$ is finite, then $\exists n:g^n = e$, let $k$ be the minimal element of $\Z$ such as $g^k = e$, then $G = \{g^0,g^1,\dots,g^{k-1}\}$ and therefore $|G|=k$. The proof is done by contradiction: suppose $\exists l,j: g^l = g^j \implies g^{l-j}=e$, but also $g^{j-l}=e$, but either $j>l$ or $l>j$. This is a contradiction and therefore it must be $|G|=k$.
	\item $|G|$ is infinite In this case $G = \{g^k:k\in\Z\}$ and $\nexists n:g^n = e$.
\end{itemize}
\subsection{Cosets and conjugacy classes}
\begin{dfn}
Let $H\leq G$ and $a\in G$, then the \textbf{left (right) coset} of $G$ is defined as
\[Ha = \{ha,h\in H\}\quad  (aH = \{ah,h\in H\})\]
\end{dfn}
Notice that the cosets are not necessarily subgroup, but two cosets of a subgroup $H$, either coincide or are disjoint. Mathematically: let $a,b\in G$ with $a\neq b$, if $\exists g\in bH,aH \implies aH = bH$. This can be proved by contradiction, suppose that $\exists g\in bH,aH$, this implies that $\exists h,h':g=bh=ah' \implies b = ah'h^{-1}$ and, because $H$ is a subgroup,
$h'h^{-1}\in H$, so $b\in aH$. Now we can take $\widetilde{h}\in H$ and let us consider $b\widetilde{h} = ah'h^{-1}\widetilde{h} \in aH$, since this holds for every $\widetilde{h}$, then $bH\subseteq aH$. The same argument can be done in order to show that also $aH \subseteq bH$. Therefore the only possibility is that $aH = bH$.\\
Furthermore, it is also clear that $\forall g\in G$, there exists $a\in G$ such that $g\in aH$. In fact, $H$ is a subgroup, so you can take $e\in H$, which implies $a=g$.\\
Considering this last two observation, we have that, for any subgroup $H\leq G$, G can be written as
\begin{equation}\label{Gdecomposition}G = \dot{\bigcup_{i\in I}}\,\, a_i H,\end{equation}
where the dot represents disjoint union. This means that we can indeed decompose our group.
\begin{thm}(Lagrange's theorem)
Let $H$ be a subgroup of $G$, then $|H|$ divides $G$, i.e. 
\[\frac{|G|}{|H|} \in \mathbb{N}\]
\end{thm}
The proof is straightforward, since $G$ can be decomposed as \eqref{Gdecomposition}, then we have that $|G| = |I|\cdot |H|$, and of course $|I|\in \mathbb{N}$.
\begin{dfn}
Let $H\leq G$ and $g\in G$, then $gHg^{-1}= \{ghg^{-1}:h\in H\}$ is a subgroup of $G$ and it is called the \textbf{subgroup conjugate} to $H$.
\end{dfn}
The proof that it is a group is easy and therefore not done. 
\begin{dfn}
Let $N\leq G$ be a subgroup, then it is called \textbf{normal group} (or normalizer) if
\[gNg^{-1} = N \quad \forall g\in G.\]
(The notation could also be $GNG^{-1}=N$)
\end{dfn}
Any group has two trivial normalizer, namely $N=\{e\}$ and $N=G$. Moreover, if $G$ is abelian, then any subgroup of $G$ is a normalizer.
\begin{dfn}
Let $G$ be a group, then the following group is called \textbf{central group} of $G$:
\[Z = \{z\in G:zg=gz\, \forall g\in G\}\]
\end{dfn}
$Z$ is always not empty, in fact it contains at least the neutral element of $G$. Furthermore, if $G$ is abelian, then $Z=G$, and for ''strong'' non abelian group $Z=\{e\}$.
\begin{dfn}
Conjugate elements $a,b\in G$ are called \textbf{conjugate} to each other if $\exists g\in G:a=gbg^{-1}$.
\end{dfn}
\end{document}
