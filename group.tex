\documentclass[12pt]{book}
\usepackage{mathtools}
\usepackage{graphicx}
\usepackage[utf8]{inputenc}
\usepackage{float}
\usepackage{tabularx}
\usepackage{chngpage}
\usepackage{amsthm}
\usepackage{mathtools}
\usepackage{amsfonts}
\usepackage{tikz}
\usepackage{amssymb}

\usepackage{wrapfig}
\usepackage{enumerate}
\usepackage{braket}
\usepackage{bbm}
\usepackage{fancyhdr}
\usepackage{emptypage}
\usepackage{braket}

\usepackage{wrapfig,lipsum,booktabs}
\input{insbox.tex}


\usetikzlibrary{decorations.markings}

\theoremstyle{plain}

%\numberwithin{equation}{section}
\newcommand{\R}{\mathbb{R}}
\newcommand{\I}{\mathbbm{1}}
\newcommand{\Z}{\mathbb{Z}}
\newcommand{\F}{\mathcal{F}}
\renewcommand{\L}{\mathcal{L}}
\newcommand{\C}{\mathbb{C}}
\renewcommand{\H}{\mathcal{H}}
\newcommand{\Sum}{\sum_{n=0}^\infty}
\newcommand{\Res}[1]{\text{Res}f(z)\Big|_{#1}}
\newcommand{\vettore}[1]{\overrightarrow{#1}}
\newcommand{\p}{\mathbf{p}}
\newcommand{\x}{\mathbf{x}}


\newtheorem{thm}{Theorem}[section]
\newtheorem{prop}[thm]{Proposizione}
\newtheorem{coro}[thm]{Corollary}
\newtheorem{lem}[thm]{Lemma}
\theoremstyle{definition}
\newtheorem{dfn}[thm]{Definition}

\theoremstyle{remark}
\newtheorem*{rmk}{Remark}
\renewcommand{\d}[2]{\frac{d #1}{d #2}} % for derivatives
\newcommand{\dd}[2]{\frac{d^2 #1}{d #2^2}} % for double derivatives
\newcommand{\pd}[2]{\frac{\partial #1}{\partial #2}} 
% for partial derivatives
\newcommand{\pdd}[2]{\frac{\partial^2 #1}{\partial #2^2}} 
% for double partial derivatives

\title{\textbf{Group theory and matrix analysis}}
\author{Marco Canteri}
\date{2017}

\def\changemargin#1#2{\list{}{\rightmargin#2\leftmargin#1}\item[]}
\let\endchangemargin=\endlist


\usepackage[a4paper, inner=1.5cm, outer=3cm, top=3cm, 
bottom=3cm, bindingoffset=1cm,headheight=110pt]{geometry} 

\pagestyle{fancy}
\fancyhf{}
\fancyhead[LE]{\leftmark}
\fancyhead[RO]{\rightmark}
\fancyfoot[C]{\thepage}
\begin{document}
\maketitle
\tableofcontents
\chapter{Group theory and representation of groups}
\section{Fundamentals of groups}
Group theory is a branch of mathematics that study \emph{Groups}, groups are useful, for instance, to describe symmetries in a system. In physics, symmetries have a central role, a problem can be simplified in presence of symmetries and some physical quantities are directly correlated to symmetries. Therefore, the study of group is indeed important in theoretical physics.
Let us start from the beginning by defining what is a group
\begin{dfn}
Let $G$ be a set and $u$ a map $u:G\times G\to G$, then the pair $(G,u)$ is called a \textbf{group} if the following conditions hold
\begin{enumerate}[i.]
\item (Associative) $g(fh) = (gf)h$;
\item (Neutral element) $\exists e\in G: eg=ge=g\quad\forall g\in G$;
\item (Inverse element) $\forall g\in G\,$ $\exists g^{-1}\in G: gg^{-1} = g^{-1}g=e$;
\end{enumerate}
notice that the map $u$ is usually represented like the standard multiplication $gh:=u(g,h)$ and very often we call $G$ a group instead of the pair $(G,u)$.
\end{dfn}
\begin{dfn}
A group $G$ is called \textbf{Abelian} if $gh=hg\quad \forall g,h\in G$.
\end{dfn}
A group can have several proprieties, one of the most important is the order of a group, which can be defined as follow
\begin{dfn}
The \textbf{order} of a group $G$ is the number of elements of $G$ (cardinality) and it is denoted as $|G|$. 
\end{dfn}
There are two cases, either the order of a group is finite (or countable infinite), and in this case the group is called discrete, or the order in not finite (uncountable infinite) and we say that the group is continuous. In order to clarify what is a group let us do some interesting examples of a group:
\begin{itemize}
\item $G = \Z_2 = \{-1,1\}$ and $u$ the standard multiplication.\\
 We need to check that this is a group, first of all the map $u$ must be closed with respect to the set, this is easy to show, in fact there are only two elements and with the standard multiplication we know that multiplying $-1$ for $1$ or vice versa will always end up with -1. The standard multiplication is also associative. The neutral element of this group is $1$ and the inverse of $g\in G$ in this group is $g$ itself. Therefore, this is a finite group.
\item $G = \{z\in \C : |z|=1\}$, $u$ standard multiplication.\\
This is also a group, in fact we can represent a complex number with a complex exponential $z=e^{i\theta}$, multiplying two complex number will only change the phase of the number, not the modulus, thus multiplication is closed. It is also associative, the neutral element is $1$ and the inverse element is the complex conjugate $zz^* = |z|^2 = 1$.
\item $G = \{e,a,a^2,a^3\}$ with $a^4=e$ and also $a^0=e$, with the map defined as $u(a^k,a^l) = a^{k+l}$.\\
It is closed, associative, the neutral element is $e$ and the inverses are: $aa^3 = e$, $a^2a^2=e$, $a^3a=e$. Therefore, this is a group of order 4.
\item $G = GL(n,\C) = \{A\in M_n(\C): \text{det}A\neq 0\}$ and $u$ is matrix multiplication. Note that $GL(n,\C)$ is usually called General Linear group and is the set of $n\times n$ invertible matrices.\\
$u$ is closed, associative, the neutral element is $\mathbbm{1}$ and the inverse is the inverse matrix $A^{-1}\in G$. Hence, this is a group.
\item $G = \{U\in M_n : U^{-1}U = \I\}$, i.e. the set of unitary matrices with the matrix multiplication.\\
The product of two unitary matrices is still unitary, so the map is closed. It is also associative, the neutral element is $\I$ and the inverse is $U^\dagger$. Therefore, it is a group.
\end{itemize}
A particular class of Group is the so called symmetric groups. Given $n$ elements $\overline{G}=\{1,2,\dots,n\}$, the symmetric group is defined as the set of permutations $\Pi:\overline{G}\to \overline{G}$, which we can write as $S_n = \{\Pi:\overline{G}\to \overline{G}: \text{where } \Pi \text{ is a bijective permutation}\}$. If $u$ is the concatenation of two permutations, then $(S_n,u)$ is a symmetric group. In fact $u$ is associative, the neutral element is the identity map, and the inverse element is $\Pi^{-1}$, which exist because permutations are bijective. Note also that the order of a symmetric group is $n!$, indeed there are $n!$ possible permutations in a set with $n$ elements. For instance, let us take $S_3$, we will denote the permutation as $\Pi_{1,2}\equiv (1,2)$, i.e the permutation that swap 1 with 2: $1\to 2,2\to 1,3\to3$, and $(1,2,3)$ as the permutation that changes every number like this: $1\to2,2\to3,3\to1$. Therefore the set $S_3$ is the following
\[S_3 = \{\I,(1,2),(2,3),(1,2,3),(1,3,2),(1,3)\}\]
there are of course $3!=6$ elements which correspond to the symmetry of a triangle, as we will see later in more detail.\\ 
We can also define a dihedral group $D_n,n\geq 3$ which denotes the orthogonal symmetries of a regular polygon, which has $n$ vertices and is centered in the origin. The vertices are denoted by $V_0,V_1,\dots,V_{n-1}\in\R^2$ in counter clockwise direction. The symmetries of a polygon are: $C_n = R_{\frac{2\pi}{n}}$, that is a counter clockwise rotation of angle $2\pi/n$; $\sigma^{(i)}$, that is a reflection with respect to the line through $V_i$ and the center. Therefore, we can write $D_n$ as
\[D_n = \{\I, C_n,C_n^2,\dots,C_n^{n-1},\sigma^{(i)},C_n\sigma^{(i)},\dots C_n^{n-1}\sigma^{(i)} \}\]
as can be easily seen, the order of this group is $|D_n| = 2n$. In the case $n=3$ we would have $D_3 = \{\I,C_3,C_3^2,\sigma^{(1)},C_3\sigma^{(1)},C_3\sigma^{(1)}\}$, notice that for instance $\sigma^{(3)} = C_3 \sigma^{(1)}$. Moreover, in this case we have that $S_3 = D_3$, in fact we can denotes every vertex of a triangle with a number and exchanging numbers is equivalent of rotating of reflecting the triangle.
\InsertBoxR{0}{
  \begin{minipage}{5.5cm}
\begin{tabular}{c|cccc}
G & $g_1$ & $g_2$ &  & $g_n$ \\
\hline 
$g_1$ &$g_1g_1$ & $g_1g_2$ &\dots& $g_1g_n$\\  
$g_2$ &$g_2g_1$ & $g_2g_2$ & & $g_2g_n$\\ 
\vdots\\
$g_n$ &$g_ng_1$ & $g_ng_2$ & & $g_ng_n$\\ 
\end{tabular}
\end{minipage}}[3]In the case of finite group, $u$ doesn't have to follow any algebraic rules. Hence, it is sometimes convenient to write $u(g_i,g_j)$ for $g_i,g_j\in G$ into a table. Due to the existence of the neutral element, there exist one row (column) which coincides with the first row (column). Furthermore, due to the existence of the inverse element, each row and each column contains all $n$ elements. In fact, suppose that there exist $i$ such that one element occurs at least twice in the $i$-th row, i.e. $\exists j\neq k : g_ig_j = g_ig_k$.  Then it also true that $g_i^{-1}g_ig_j = g_i^{-1}g_ig_k \implies g_j = g_k$ which is a contradiction. Therefore, all the $n$ elements of the $i$-th row must be different, that is each row contains all $n$ element of $G$. The same argument can be done with a column. From this we can say that the multiplication of $g_i$ corresponds to a permutation $\Pi_{g_i}$ of the $n$ elements of $G$.
\section{Subgroups}
A subgroup of a group is basically a subset of a group which is itself a group, or more formally
\begin{dfn}
Let $G$ be a group, then $H\subset G$ is a \textbf{subgroup} of $G$ if
\begin{enumerate}[i.]
	\item $e_g \in H$
	\item $\forall g,h \in H,$ $gh\in H$
	\item $\forall g\in H,$ $g^{-1}\in H$
\end{enumerate}
\end{dfn}
To distinguish between subgroup and subset, there is a different notation, so if $H$ is a subgroup of $G$ we write $H\leq G$, while if $H$ is only a subset the notation is the standard $H\subset G$. A couple of examples of subgroups, which are relevant for their applications in physics, are
\begin{itemize}
\item $SL(n,\C) = \{A\in GL(n,\C): |A| = 1\}\leq GL(n,\C)$, $SL$ is called the special linear group and it is a subgroup of the general linear group of matrices which have determinant equals to one.
\item $SO(n) = \{A\in SL(n,\R) : A^\dagger A = \I\}\leq SL(n,\R)\leq GL(n,\C)$ 
\item $U(n) = \{A\in GL(n,\C):A^\dagger A= \I \} \leq GL(n,\C)$
\end{itemize}
A property of subgroup is that if we take $H_i\leq G$ subgroups, with an index $i\in I$ the the intersection of the these subgroup is still a subgroup, that is
\[H = \bigcap_{i\in I} H_i \leq G.\]
The proof is straightforward, we only need to check if the intersection has the proprieties of a group. Associative propriety is obvious. Since $H_i$ are a group, the neutral element is in each of them, so the neutral element is also in the intersection. The same argument goes for the inverse, $g\in H_i \implies g^{-1}\in H_i$, which means that $\forall g\in H \implies g^{-1}\in H$. And lastly with the same logic as before if $g,h\in H_i\implies gh \in H_i$ then $gh\in H$ if $g,h\in H$.\\
As already said, subsets and subgroup are different, but it is also possible to generate a subgroup from a subset. Let $G$ be a group and $E\subset G$ a subset, then the subgroup generated by $E$ is
\[\braket{E} = \bigcap_{\substack{E\subseteq H\\H\leq G}} H,\]
which is the smallest subgroup that contains $E$. It can be easily computed by taking the elements of $E$, all their inverse and all their products. For instance, let us consider the set $E = \{C_3\}$, the inverse of this element is $C_3^2$ and the neutral element is $\I$. Therefore the subgroup generated by $E$ is
\[\braket{E} = \{C_3,C_3^2,\I\}.\]
It is also easy to show that the following symmetric group can be written as $S_3 = \braket{\{C_3,\sigma^{(1)}\}}$, or that $\Z$ can be generated by $\Z = \braket{\{1\}}=\braket{\{-1\}}$, where the map $u$ is the addition. Last example shows also that a generate set of a subgroup is not necessarily unique. Moreover it is called cyclic, the general definition is the following
\begin{dfn}
A group is called \textbf{cyclic} if it is generated by a single element.
\end{dfn}
Cyclic groups $G$ can be divided in two classes:
\begin{itemize}
	\item If $|G|$ is finite, then $\exists n:g^n = e$, let $k$ be the minimal element of $\Z$ such as $g^k = e$, then $G = \{g^0,g^1,\dots,g^{k-1}\}$ and therefore $|G|=k$. The proof is done by contradiction: suppose $\exists l,j: g^l = g^j \implies g^{l-j}=e$, but also $g^{j-l}=e$, but either $j>l$ or $l>j$. This is a contradiction and therefore it must be $|G|=k$.
	\item $|G|$ is infinite In this case $G = \{g^k:k\in\Z\}$ and $\nexists n:g^n = e$.
\end{itemize}
\subsection{Cosets and conjugacy classes}
\begin{dfn}
Let $H\leq G$ and $a\in G$, then the \textbf{left (right) coset} of $G$ is defined as
\[Ha = \{ha,h\in H\}\quad  (aH = \{ah,h\in H\})\]
\end{dfn}
Notice that the cosets are not necessarily subgroup, but two cosets of a subgroup $H$, either coincide or are disjoint. Mathematically: let $a,b\in G$ with $a\neq b$, if $\exists g\in bH,aH \implies aH = bH$. This can be proved by contradiction, suppose that $\exists g\in bH,aH$, this implies that $\exists h,h':g=bh=ah' \implies b = ah'h^{-1}$ and, because $H$ is a subgroup,
$h'h^{-1}\in H$, so $b\in aH$. Now we can take $\widetilde{h}\in H$ and let us consider $b\widetilde{h} = ah'h^{-1}\widetilde{h} \in aH$, since this holds for every $\widetilde{h}$, then $bH\subseteq aH$. The same argument can be done in order to show that also $aH \subseteq bH$. Therefore the only possibility is that $aH = bH$.\\
Furthermore, it is also clear that $\forall g\in G$, there exists $a\in G$ such that $g\in aH$. In fact, $H$ is a subgroup, so you can take $e\in H$, which implies $a=g$.\\
Considering this last two observation, we have that, for any subgroup $H\leq G$, G can be written as
\begin{equation}\label{Gdecomposition}G = \dot{\bigcup_{i\in I}}\,\, a_i H,\end{equation}
where the dot represents disjoint union. This means that we can indeed decompose our group.
\begin{thm}(Lagrange's theorem)
Let $H$ be a subgroup of $G$, then $|H|$ divides $G$, i.e. 
\[\frac{|G|}{|H|} \in \mathbb{N}\]
\end{thm}
The proof is straightforward, since $G$ can be decomposed as \eqref{Gdecomposition}, then we have that $|G| = |I|\cdot |H|$, and of course $|I|\in \mathbb{N}$.
\begin{dfn}
Let $H\leq G$ and $g\in G$, then $gHg^{-1}= \{ghg^{-1}:h\in H\}$ is a subgroup of $G$ and it is called the \textbf{subgroup conjugate} to $H$.
\end{dfn}
The proof that it is a group is easy and therefore not done. 
\begin{dfn}
Let $N\leq G$ be a subgroup, then it is called \textbf{normal group} (or normalizer) if
\[gNg^{-1} = N \quad \forall g\in G.\]
(The notation could also be $GNG^{-1}=N$)
\end{dfn}
Any group has two trivial normalizer, namely $N=\{e\}$ and $N=G$. Moreover, if $G$ is abelian, then any subgroup of $G$ is a normalizer.
\begin{dfn}
Let $G$ be a group, then the following group is called \textbf{central group} of $G$:
\[Z = \{z\in G:zg=gz\, \forall g\in G\}\]
\end{dfn}
$Z$ is always not empty, in fact it contains at least the neutral element of $G$. Furthermore, if $G$ is abelian, then $Z=G$, and for ''strong'' non abelian group $Z=\{e\}$.
\begin{dfn}
Conjugate elements $a,b\in G$ are called \textbf{conjugate} to each other if $\exists g\in G:a=gbg^{-1}$.
\end{dfn}
Before continuing, let us recall what a equivalent relation is. It is denoted by $\sim$ and it satisfies the following proprieties
\begin{enumerate}
	\item (reflex) $a \sim a$
	\item (symmetric) $a\sim b \iff b\sim a$
	\item (transitive) $a\sim b$ and $b\sim c$ $\implies a\sim c$
\end{enumerate}
Let us go back to conjugation, it is easy to prove that conjugation defines an equivalent relation. In fact:
\begin{itemize}
\item $a\sim a$, we can choose $g=e$, and we have $a=a$
\item $a\sim b \implies \exists g\in G: a=gbg^{-1} \implies b = g^{-1}ag \implies b\sim a$
\item $a=g_1bg^{-1}_1,b=g_2cg_2^{-1}\implies a = g_1g_2cg_1^{-1}g_2^{-1}\implies a = g_3cg_3$
\end{itemize}
\begin{dfn}
The \textbf{conjugacy class} of an element $a\in G$ is defined as
\[K_a = \{gag^{-1}:g\in G\}\]
\end{dfn} 
If we choose $g=e$, we immediately find that $a\in K_a$.
\begin{thm}
Any group can be decomposed in disjoint conjugacy classes $K_i$. Let $n(K_i)$ denote the cardinality of $K_i$, then it holds that $|G| = \sum_i n(K_i)$
\end{thm}
The proof is left to the reader as exercise.
\begin{thm}
The number of elements of a conjugacy class $n(K_i)$ divides $|G|$
\end{thm} 
The proof is left to the reader as exercise.
\begin{coro}
If $|G|$ is prime, then $n(K_i)=1\, \forall K_i \implies G$ is abelian.
\end{coro}
\section{Homomorphism and isomorphism}
Before jumping into representation theory is necessary to define and understand what homomorphism and isomorphism are in group theory.
\begin{dfn}
Let $G,H$ be groups, then a function $f:G\to H$ is called \textbf{group homomorphism} (GH) if
\[f(g_1g_2) = f(g_1)f(g_2)\in H\quad \forall g_1,g_2\in G\]
\end{dfn}
\begin{dfn}
Let $G,H$ be groups, then a function $f:G\to H$ is called \textbf{group isomorphism} (GI) if it is bijective. And we write $G\simeq H$
\end{dfn}
Furthermore, the following hold
\begin{enumerate}
	\item $f(e_G) = e_H$, in fact $f(g) = f(e_Gg)=f(e_G)f(g)\,\forall g$, hence $f(e_G)$ must be $e_H$.
	\item $f(g^{-1}) = f(g)^{-1}$, in fact $f(gg^{-1})=f(e_G) = e_H \implies f(g)f(g^{-1}) = e_H \implies f(g^{-1}) = f(g)^{-1}$.
	\item if $f$ is a GI, then $f^{-1}$ is a GI.
\end{enumerate}
Let us do some examples with of homomorphisms and isomorphisms.
\begin{itemize}
\item The determinant of a matrix $\text{det}:GL(n,\C) \to \C$ is a GH, in fact from the well known propriety of determinant $|AB| = |A||B|$.
\item Let $G$ be a group and $g\in G$, the function $f:\Z \to G,k\mapsto g^k$ is a GH. In fact, $f(k_1k_2) = g^{k_1+k_2} = g^{k_1}g^{k_2} = f(k_1)f(k_2)$. (Recall that for the group $\Z$ we can choose as map the addition)
\item the exponential $\R\to \R^+, x\mapsto e^x$ is a GI.
\end{itemize}
\begin{dfn}
Let $G,H$ be groups and $f:G\to H$ a GH, then we define the \textbf{kernel} of $f$ as
\[\text{Ker}(f) = \{g\in G:f(g)=e_H\}\]
Moreover, we define the \textbf{range} of $f$ as
\[R(f) = \{f(a):a\in G\}\]
\end{dfn}
From this definitions it is easy to derive that the kernel is a subgroup of $G$, while the range is a subgroup of $H$. Indeed, if $g_1,g_2\in \text{Ker}(f)\implies f(g_1)=f(g_2)=e_H$, but due to the fact that $f$ is a GH, it is also true that $f(g_1g_2) = f(g_1)f(g_2)=e_H$. Therefore, $g_1g_2\in \text{Ker}(f)$. Furthermore, $e_G\in \text{Ker}(f)$, in fact $f(e_G) ) e_H$. And finally if $g\in \text{Ker}(f)\implies f(g^{-1}) = f(g)^{-1} = e_H^{-1} = e_H$, therefore $g^{-1}\in\text{Ker}(f)$. This prove the properties of a group and thus $\text{Ker}(f)\leq G$. Similarly can be demonstrated also that $R(f)\leq H$. Moreover, if $f$ is also injective, then $\text{Ker}(f) = \{e_G\}$ and if $f$ is subjective, then $R(f) = H$. \\
In the first section of this chapter we did some examples of groups, in particular we studied the symmetric group, Its central importance in group theory is due to the following theorem that connects finite group to symmetric group through an isomorphic map. Furthermore, groups can be represented by matrices via an homomorphic map, as we will see later.
\begin{thm}
(Theorem of Cayley for finite groups) Every finite group of order $n$ is isomorphic to a subgroup of the symmetric group $S_n$.
\end{thm}
The proof is based on the representation of a group with a group table. Each column and each row contains all elements of $G$. Hence, multiplication of a row or a columns with an element of $G$ corresponds to a permutation of the element of $G$. $g\mapsto \Pi_g, G\to H\leq S_n$ is a GI.
\section{Representation of groups}
The elements of a group can be very abstract object with which it could be difficult to work with. However, it is possible to map group into matrices that can be manipulated more easily. This mapping is the aim of representation theory, it can be proved that a group has infinity many representation. Let us start as usual, by defining a representation.
\begin{dfn}
A representation $D$ of a group $G$ is a GH $D:G\to GL_k(V)$, where $GL_k(V)$ is the group of general linear function over a vector space $V$. 
\end{dfn} 
It can be showed that $GL_k(V)$ is isomorphic to $GL_n(\C)$, which are invertible matrices. Therefore, $D$ maps the group $G$ into invertible $n\times n$ matrices, where $n= \text{dim}(n)$.
Some remarks, $V$ is called carrier (representation) space and $\text{dim}(n)$ is called the dimension of the representation. As notation we use $D_G$ or $D(G)$ for the representation and in the rest of this work we will consider only finite dimensional representation. \\
Every group $G$ has a \emph{trivial representation}, that is $D:G\to \C \simeq GL_1,g\mapsto 1$. It easy to check that this is indeed a representation, however all the information about the group is lost. This trivial representation is one dimensional, but it is also possible to define a trivial representation in higher dimension (cazzo, molto utile).
\begin{dfn}
A representation $D$ is called \textbf{faithful} if $D$ is injective.
\end{dfn}
As already said, a group can have different representation, but two representation  $D_1:G\to GL(V),D_2:G\to GL(W)$ are called equivalent if there exists a mapping $f:W\to V$ which is an isomorphism
\[D_1(g) = f\circ D_2(g)\circ f^{-1}\quad \forall g\in G.\]
Before looking at some example, notice that, since $D$ is a GH, it holds that $D(g_1g_2)=D(g_1)D(g_2)$ for any representation $D$. Therefore, $D(g)$ can be computed from $D(g_i)$, where $g_i$ are the generating element, $G =\braket{\{g_i\}}$. For instance, let us consider $S_3 \simeq D_3 = \braket{\{\sigma^1,C_3\}}$, we can compute $D_3(C_3^2) = D_3(C_3)D_3(C_3)$, where $D_3(C_3)$ is known. The same for $D_3(C_3^{-1})=D_3(C_3)^{-1}$. \\
Let us go through some instances of representation of the group $S_3$.
\begin{itemize}
\item A two dimensional representation of $S_3$ could be the following $D_2:S_3 \to GL_2$:
\[C_3 \mapsto \begin{pmatrix}
  \cos\left(\frac{2\pi}{3}\right) & \sin\left(\frac{2\pi}{3}\right) \\
  -\sin\left(\frac{2\pi}{3}\right) & \cos\left(\frac{2\pi}{3}\right) 
 \end{pmatrix} \qquad \sigma^1 =  \begin{pmatrix}
 -1& 0 \\
0 & 1
 \end{pmatrix}.\]
This representation is also faithful.
\item A one dimensional representation is the map
\[C_3 \mapsto 1\qquad \sigma^1 \mapsto 1,\]
which is a trivial representation.
\item Another non trivial one dimensional representation is
\[C_3\mapsto 1 \qquad \sigma^1 \mapsto -1.\]
The number are not chosen randomly, but they are the determinant of previous matrices. This representation is not faithful and can be generalized. In fact, it can be demonstrated that if $D:F\to GL_n$ is a representation of $G$, then $D:G\to \C, g\mapsto \text{det}(D(g))$ is a one dimensional representation of $G$.
\item A three dimensional representation can be done by adding one row and one columns to the two dimensional representation, as follow
\[C_3 \mapsto \begin{pmatrix}
  \cos\left(\frac{2\pi}{3}\right) & \sin\left(\frac{2\pi}{3}\right) & 0\\
  -\sin\left(\frac{2\pi}{3}\right) & \cos\left(\frac{2\pi}{3}\right) & 0\\
  0 & 0 &1 
 \end{pmatrix} \qquad \sigma^1 =  \begin{pmatrix}
 -1& 0 & 0\\
0 & 1 & 0\\
0 &0 &1
 \end{pmatrix}.\]
This can be also done for higher dimensional representation, but they are not very interesting. However, there are representation called irreducible with which every representation can be constructed, as we will see later.
\item There is a more interesting three dimensional representation of $S_3$. First let us recall that the action of a group with representation $D_3:S_3\to GL(\C^3)$ can be written using the standard basis $\{\mathbf{e_i}\}$ of $\C^3$, in fact $D_3(g)\mathbf{e_i} =\mathbf{e_g}$
\[D_3(g_1) = \begin{pmatrix}
  0 & 1 & 0\\
  1 & 0 & 0\\
  0 & 0 & 1 
 \end{pmatrix} \quad D_3(g_2) = \begin{pmatrix}
  0 & 0 & 1\\
  0 & 1 & 0\\
  1 & 0 &0 
 \end{pmatrix}\quad D_3(g_3) = \begin{pmatrix}
  1 & 0 & 0\\
  0 & 0 & 1\\
  0 & 1 &0 
 \end{pmatrix}\]
 \[D_3(g_4) = \begin{pmatrix}
  0 & 0 & 1\\
  1 & 0 & 0\\
  0 & 1 &0 
 \end{pmatrix}\quad D_3(g_5) = \begin{pmatrix}
  0 & 1 & 0\\
  0 & 0 & 1\\
  1 & 0 &0 
 \end{pmatrix},\]
 and the identity matrix which is the neutral element. In order to show that this is indeed a representation, it has to fulfil $D_3(g_ig_j) = D_3(g_i)D_3(g_j)$, i.e. the representation is a GO. This can be done easily with the Dirac notation $D(g_ig_j) = \sum_i \ket{e_{g_i}}\bra{e_{g_i}}$. This last example is called regular representation, which we are going to focus on in the next section.
\end{itemize}
\subsection{Regular representations}
A regular representation is defined for any finite group, and if $|G|=n$, then the representation $R(p)\in M_{n,n}(\R)$. Moreover, the regular representation contains in a sense all the representation.
\begin{dfn}
Let $G$ be a group, then a \textbf{regular representation} $R:G\to GL_n(\R)$ is defined as follow
\[R(g_i)_{kj} = \begin{cases}
1\qquad \text{if}\,\,\, g_kg_j^{-1} = g_i\\
0 \qquad \text{else}
\end{cases}\]
\end{dfn}
\InsertBoxR{0}{
  \begin{minipage}{5.5cm}
\begin{tabular}{c|cccc}
G & $g_1$ & $g_2$ &  & $g_n$\\
\midrule
$g_1^{-1}$ &$e$ & $g_1^{-1}g_2$ &\dots& $g_1^{-1}g_n$\\  [0.05cm]
$g_2^{-1}$ &$g_2^{-1}g_1$ & $e$ & & $g_2^{-1}g_n$\\ [0.05cm]
$\vdots$ & & & $\ddots$\\[0.05cm]
$g_n^{-1}$ &$g_n^{-1}g_1$ & $g_n^{-1}g_2$ & & $e$\\ [0.05cm]
\end{tabular}
\end{minipage}}[3]This definition can be better understood by considering the group table. Since the order of the elements is not relevant, we can sort the table such that, on the diagonal, there is only the neutral element. $R(g_i)$ is then defined as the $n\times n$ matrix that we obtain from the table by replacing $g_i$ with one, and $g_j\neq g_i$ with zeros on each row.
It is possible to prove that $R$ is a representation, this is left as an exercise for the reader.
\begin{dfn}
Let $D:G\to GL(V)$ be a representation, then $D$ is called \textbf{unitary} if
\[D(g)^\dagger = D(g)^{-1} = D(g^{-1})\quad \forall g \in G\]
\end{dfn}
Let us consider two representation $D_1,D_2$ of $G$, then $D_1\oplus D_2:G \to GL(V_1\oplus V_2),g\mapsto D_1(g)\oplus D_2(g)$ is a representation. Therefore, it is possible to build representation. This also means that there exist infinitely many representation, as we are going to see, for finite groups only very few representations are necessary to construct all of them. Notice that if $D_1\oplus D_2$ is a representation, also $S(D_1\oplus D_2)S^{-1}$ is a representation for any $S\in GL(\text{dim}(V_1)+\text{dim}(V_2))$. 
\subsection{Irreducible representations}
We have already remarked that a representation can be build with others. However, there are some representations that constitute the basic block for building other representations, and cannot be obtained from other representation. This representations are called irreducible representations. Before jumping in their definition, we need to discuss a particular class of spaces.
\begin{dfn}
Let $(D,V)$ be a representation, then a subspace $W\subseteq V$ is called \textbf{invariant subspace} if
\begin{enumerate}[i.]
	\item $D(g)W\subseteq W, \forall g\in G$, i.e. $D(g)w\in W$ for $w\in W$.
	\item $W\neq \{0\}$ and $W\neq V$.
\end{enumerate}
\end{dfn}
\begin{dfn}
A representation $(D,V)$ is called \textbf{irreducible} if there exist no non-trivial invariant subspace, i.e. there exist no invariant subspace $W$ with $W\neq\{0\}$ and $W\neq V$. Otherwise $(D,V)$ is called \textbf{reducible}.
\end{dfn}
Invariant subspaces are important because we can actually work inside them, in fact, let $W$ be an invariant subspace of a representation $(D,V)$, then $D$ restricted to $W$ is a representation
\[D\big|_W:G\to GL(W)\qquad g\mapsto D\big|_W(g).\]
\begin{dfn}
A representation $(D,V)$ is called \textbf{completely reducible} if there exist invariant subspaces $V_i\subseteq V$ such that $V_1\oplus V_2\oplus\dots\oplus V_n = V$. If  $(D\big|_{V_i},V_i)$ are irreducible, then $V_1\oplus V_2\oplus\dots\oplus V_n = V$ is called \textbf{decomposition} into irreducible representations. Furthermore, we can write our representation as
\[D = \bigoplus_{i}D\big|_{V_i}\]
\end{dfn}
Notice that not any representation is completely reducible, for instance let us consider the following representation
\[D(\R,+)\to GL_2\quad x\mapsto \begin{pmatrix}
  1 & x\\
  0 & 1 
 \end{pmatrix} \]
If we apply our matrix on the vector $(1,0)^{T}$ we obtain $(1,0)^{T}$. Therefore, this vector is invariant, but there exist no $v$ such that $D(x)v$ is proportional to $v$, that is the matrix is not diagonalizable, thus we cannot write $\R$ as a direct sum of two invariant subspaces. Hence $D$ is not completely reducible.\\
Nevertheless, for this example we considered a non finite group, but we will see that for a finite group, any representation is completely reducible. \\
A relevant case for physics is when $V$ is a Hilbert space, and therefore there exists a scalar product on such space, therefore the orthogonal complement $W^\perp$ is well defined and can be used.
\begin{lem}\label{lemmautile}
A representation $(D,V)$ is completely reducible if and only if for any invariant subspace $W$, there exist an invariant subspace $W'$ such that $W\oplus W' = V$
\end{lem}
\hspace{-1.4em}\textbf{Proof:}\\
$\implies$) $(D,V)$ is completely reducible $\implies V= V_1\oplus V_2\oplus \dots V_n$, where $V_i$ are invariant subspaces and does not contain any non-trivial invariant subspace. If $W$ is an invariant subspace, then $W\cap V_i$ is either $\{0\}$ or $V_i$ this implies that $W = \bigoplus_{i\in I} V_i$ for some indexes $I$, thus $W^\perp = \bigoplus_{i\notin I} V_i \implies W^\perp$ is invariant.\\
$\impliedby$) The proof is by induction on the dimension of $V$. If $\text{dim}(V) = 1$, the statement is obvious $V= V_1$.Now let us assume that it is true for $\text{dim}(V) = d$, and we prove that it is true for $\text{dim}(V) = d+1$. $V$ is either irreducible or exists an invariant subspace $W\subset V$ with $1\leq \text{dim}(W)\leq d$. Per assumption $\exists W' : V = W\oplus W'$ and $W'$ is irreducible and also $\text{dim}(W') \leq d$ which implies that $W,W'$ are completely irreducible, i.e. $V$ is completely reducible. \hfill\ensuremath{\square}\\
As example let us study the symmetric group $S_3$, recall the three dimensional representation $D_3(g)$ which can be written as $D_3(g)e_i = e_{g_i}$. Our goal is to find an invariant subspace $W\subset \R^3$ such that $D_g W = W$ $\forall g\in G$. For instance
\[W_1 = \text{span}\left\{\begin{pmatrix}1\\1\\1\end{pmatrix}\right\}\]
is an invariant subspace with dimension one, in fact applying a permutation to this space does not change it. Another is 
\[W_2 = \left\{\begin{pmatrix}x_1\\x_2\\x_3\end{pmatrix}:\sum_i x_i = 0\right\}.\]
This is a two dimensional subspace. The point now is: can we reduce further this space? If the answer is no we decomposed $D_3$, otherwise we can further decompose it. The answer is no, there is not, and we can show it. Consider an hypothetical invariant subspace $\overline{W}$ of $W$ that is non trivial. The dimension of $\overline{W}$ must be one and $D_g\overline{W}\propto \overline{W}$, for instance we we take the permutation $2\to1$
\[D_g \begin{pmatrix}w_1\\w_2\\w_3\end{pmatrix} \propto \begin{pmatrix}w_2\\w_1\\w_3\end{pmatrix}\quad \forall g\in G.\]
This must holds for every permutation. Therefore, $\omega_1 \propto \omega_2\propto \omega_3$ and $\sum_i \omega_i = 0$. The only solution for $\omega_i$ is $\omega_i =0$ that is a trivial space. In conclusion, $W_1$ e $W_2$ are the only invariant subspaces which implies that $D_3$ is completely reducible with $V = W_1 \oplus W_2$ and more important $D = D\big|_{W_1} \oplus D\big|_{W_2}$. We will return on this example in the next section.
\section{Finite groups}
\begin{thm}
Any irreducible representation of a finite group is finite dimension.
\end{thm}
\hspace{-1.4em}\textbf{Proof:}\\
The proof can be done by contradiction. Suppose that $V$ is infinite dimensional and irreducible, then we can define the space $W_a = \{D_g \ket{a}:g\in G,a\in V\}$. $W_a$ is finite because $G$ is finite and it is invariant which is a contradiction of the assumption.\hfill\ensuremath{\square}
\begin{thm}
Any representation of finite group is equivalent to a unitary representation, i.e.
\[\forall (D,V),\forall g\in G \,\,\, \exists S:SDS^{-1} = \widetilde{D}_g \,\,\,\text{and}\,\,\, \widetilde{D}_g^{-1} = \widetilde{D}_g^\dagger  \]
\end{thm}
\hspace{-1.4em}\textbf{Proof:}\\
Let $D$ be a representation of $G$, we define the positive\footnote{in fact $\braket{\psi|D^\dagger_gD_g|\psi}=\|\psi\|^2$, and sum of positive terms is always positive.} operator $T = \displaystyle\sum_{g\in G}D^\dagger_gD_g$. Then we show the following
\begin{equation}\label{questarobaqui}
D_h^\dagger T D_h = D_h^\dagger \sum_{g\in G} D_g^\dagger D_g D_h = \sum_{g\in G} D_h^\dagger D_g^\dagger D_gD_h = \sum_{g\in G} D_g^\dagger D_g  \equiv T,
\end{equation}
whereas in the last step we used the fact that $D_gD_h = D_{gh}$, and we have already showed that $gh$ is just a permutation of the elements, therefore the sum over every term $D_{gh}$ is equivalent as the sum $D_g$ for the commutativity of the sum. Obviously, equation \eqref{questarobaqui} holds for every $h\in G$. Now we define $S =T^{\frac{1}{2}}$, which is well defined because $T$ is positive\footnote{$T= T^\dagger\implies \exists u,d:T=udu^\dagger$, where $d$ is diagonal. That is $T^{1/2}=U\sqrt{D}U^\dagger$}, and we rewrite equation \eqref{questarobaqui}
as $D_h^\dagger S^2 D_h = S^2$, which can be rewritten as $D_h^\dagger S = S^2 D_h^{-1}S^{-1}$. Multiplying this by $S^{-1}$ leads to $S^{-1}D_h^\dagger S = SD_h^{-1}S^{-1}$ and finally taking the inverse $(S^{-1}D_h^\dagger S)^{-1} = S^{-1}D_hS$, in the right hand side we can write instead of $S$, $S^\dagger$, since they are the same and we obtain the final result
$(S^{-1}D_h^\dagger S)^{-1} = (S^{-1}D_h^\dagger S)^\dagger$ \hfill(?:/)\ensuremath{\square}
\begin{thm}
Any unitary representation of a finite group is completely reducible
\end{thm}
\hspace{-1.4em}\textbf{Proof:}\\
Let $W$ be invariant, and let $a\in W^\perp,b\in W,p\in G$, then $\braket{a|D_g|b} = \braket{D_g^\dagger a|b} = \braket{D_{g^{-1}}a|b}\implies D_g W^\dagger \subseteq W^\dagger$. 
According to lemma \ref{lemmautile} a representation $(D,V)$ is completely reducible if and only if for all invariant subspace $W$ also $W^\perp$ is invariant. \hfill(?:/)\ensuremath{\square}\\
In summary, we showed the following propositions
\begin{enumerate}
	\item Any representation of finite group is finite
	\item Any representation of finite group is equivalent to an unitary group
	\item Any unitary representation is completely reducible
\end{enumerate}
as corollary of these we have the most important result
\begin{center}\fbox{\begin{minipage}{\dimexpr.5\textwidth-2\fboxsep-2\fboxrule\relax}
\centering
Any representation of finite group \\is completely reducible
\end{minipage}}
\end{center}
That is, if $G$ is finite, let us consider $D:G\to GL(V)$, then 
\[\exists S\in GL(V): D_g = S\begin{pmatrix}D_1(g)\\&D_2(g)\\&&\ddots\end{pmatrix}S^{-1},\]
where $D_i$ are irreducible representations.\\
Let us continue with the usual example of the representation of $S_3$. We have already showed how it can be decomposed into two invariant space $W_1$ and $W_2$, now let us try to find the representations restricted to those spaces. This can be done projecting the operator into the subspace
\[D_3(g)\big|_{W_1} = P_{W_1}D_3(g)P_{W_1} = \ket{w_1}\bra{w_1}D_3(G)\ket{w_1}\bra{w_1} = \ket{w_1}\bra{w_1}\quad \forall g\in G.\]
In fact a permutation over a state with the same components does not change anything. Hence $D_3(g)\big|_{W_1}$ is the trivial representation $D_3(g)\big|_{W_1}:G\to 1$. More interesting is of course the space $W_2$, consider a base of this space, for instance
\[v_1 = \frac{1}{\sqrt{2}}\begin{pmatrix}1\\0\\-1\end{pmatrix}\qquad v_2 = \frac{1}{\sqrt{6}}\begin{pmatrix}-1\\2\\-1\end{pmatrix}.\]
Then the projector is $P_{W_2} = \ket{v_1}\bra{v_1} + \ket{v_2}\bra{v_2}$, hence
\[D_3(g)\big|_{W_2} = P_{W_2}D_3(g)P_{W_2} = \sum_{ij} \ket{v_i}\bra{v_i}D_3(g)\ket{v_j}\bra{v_j},\]
which can be calculated by hand or every permutation. A faster way is to consider The decomposition of $D$ as
\[D_s(g) = S \begin{pmatrix}D_3(g)\big|_{W_1}&0\\0& D_3(g)\big|_{W_2} \end{pmatrix}S^{-1} =S \begin{pmatrix}1&0\\0& D_3(g)\big|_{W_2} \end{pmatrix}S^{-1}.\]
$D_3(g)\big|_{W_2} $ must be a two dimensional representation of $S_3$ which we have already calculated in previous example.
 \end{document}